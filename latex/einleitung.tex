\chapter{Einleitung}
\label{cha:einleitung}
In diesem Abschnitt wird 'Abyss' kurz vorgestellt, dabei wird vor allem auf 
die Motivation und das Ziel der Applikation eingegangen.

% Abschnitt: Problemstellung
\section{Problemstellung}
\label{sec:einleitung:problemstellung}
Super Mario Bros. \cite{mario} ist wohl eines der bekanntesten Jump'n'Run Spiele und
der Grundstein für einige weitere Platformer. Trotz der großen Anzahl an Spielen,
die zurzeit auf dem Markt erhältlich sind, ist wohl kaum eines so erfolgreich gewesen
wie die, in den der kleine rote Klemptner Mario durch verschiedene Welten rennt. \newline
Da Nintendo erst spät damit angefangen hat die App-Stores für sich zu entdecken 
fehlte vielen Spielern lange Zeit ein guter Platformer für das Smartphone, 
der nicht langweilig wird.


% Abschnitt: Zielsetzung
\section{Zielsetzung}
\label{sec:einleitung:zielsetzung}
Das Ziel ist ein Mobile Game zu entwickeln, dass eine möglichst intuitive Bedienung
hat. Zudem soll das Spiel vielfältig in seiner Funktionalität sein, sodass es 
über einen längeren Zeitraum das Interesse vom Spieler halten kann. \newline
Das beinhaltet nicht nur die grundlegenden Spielfunktionen, sondern auch eine 
Multiplayerfunktion, die im Zuge der Entwicklung umgesetzt werden soll. 
Die Spielfunktionen sollen in verschiedenen Leveln umgesetzt werden, die sich 
auch im Grunddesign unterscheiden.

% Abschnitt: Struktur der Arbeit
\section{Struktur der Arbeit}
\label{sec:einleitung:struktur}
Nach dieser kurzen Einleitung sollen die Grundlagen für die Arbeit erläutert werden.
Anschließend sollen die Anforderungen herausgearbeitet werden, die später zu einem Konzept 
ergänzt werden sollen. Anschließend wird beschrieben, wie das Konzept umgesetzt wurde, bevor
die Arbeit in einer Zusammenfassung und einem Ausblick beendet wird.