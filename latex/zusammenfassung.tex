\chapter{Zusammenfassung \& Ausblick}
\label{cha:zusammenfassungAusblick}
In diesem letzten Kapitel wird die Arbeit nochmal zusammengefasst um anschließend einen Ausblick für die Zukunft zu geben.

% Abschnitt: Zusammenfassung
\section{Zusammenfassung}
\label{sec:grundlagen:zusammenfassung}
Da Unity als Engine schon viele vordefinierte Funktionen anbietet, darunter zum Beispiel die Collision Detection, ist das Arbeiten zunächst ungewohnt im Gegensatz zur Arbeit mit einer Entwicklungsumgebung wie Android Studio. Ein Grund dafür ist, das ein großer Anteil nicht über Programmcode, sondern über die GUI-Oberfläche die Unity zur Verfügung stellt realisiert wird. Der Inspector des Editors lässt die Componenten der Gameobjects mit den jeweiligen Attributen schnell verändern. Dadurch ist das Grundgerüst des Spiels zwar nur sehr aufwendig zu realisieren, aber später auch sehr wandlungsfähig und kann im Nachhinein nach belieben im Detail angepasst werden. 

Nach dem Fertigstellen der App kann durch die gute Kompatibilität von Unity mit sehr vielen Plattformen das Spiel nicht nur auf Android, sondern auf weiteren Plattformen verwendet werden. Dadurch, und durch die Verwendung von Photon, wird unsere Abyss-Anwendung zu einem \textit{Crossplatformgame}.

Das Verwenden von eigenen UI-Elementen sollte man als Design-Anfänger outsourcen, da das Erstellen von Bilder, die visuell ansprechend sein sollen, eine künsterische Ader erfordert. Das gleiche gilt für die Erstellung von eigenen Soundeffekten und Hintergrundmusik. Man kann durch die Verwendung von eigenen Materialien das Spiel im hohem Maß individualisieren, doch ist der Aufwand, der hinter dieser Arbeit steckt, nicht zu unterschätzen. 

Das Projektmanagement der Unity-Anwendung ist mit Git fehleranfällig. Erstellt man jedoch für jedes Problem einen eigenen Branch, und versucht diese und damit das Mergen so klein wie möglich zu halten, können die enstehenden Probleme minimal gehalten werden.

% Abschnitt: Ausblick
\section{Ausblick}
\label{sec:grundlagen:ausblick}
Nachdem das Grundgerüst des Spiels mit den selbstentworfenen Funktionen und Assets erstellt wurde, ist das Spiel schnell und simpel erweiterbar. Durch das Verwenden von Prefabs und den vorhandenen Tilemaps können in kürze Spielflächen erstellt und Hindernisse hinzugefügt werden. Um neue Spielobjekte zu erstellen, können die vorhandenen Skripte angefügt und durch neue, aber simple Skripte erweitert werden. Dadurch kann das Spiel nicht nur an Leveln, sondern auch an Spielkomplexität wachsen.