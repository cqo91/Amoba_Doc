\documentclass[numbers=noenddot, openany]{thesis}

% hier namen etc. einsetzen
\fullname{Lukas Hennig\\Felix Rottler\\Natalie Spister}
\headline{Advanced Mobile Business Applications\\Dokumentation}
\jahr{2018}
\gutachterA{Marc Schickler}
\typ{Anwendungsfach }
\fakultaet{Ingenieurwissenschaften, Informatik und \\Psychologie}
\institut{Institut für Datenbanken und Informationssysteme}

\hypersetup{%
	pdftitle=\pdfescapestring{\thetitel},
	pdfauthor={\thefullname},
 	pdfsubject={\thetyp},
}


\usepackage{graphicx}
\usepackage{caption}
\usepackage{subcaption}
\usepackage{float}


% trennungsregeln
\hyphenation{Sil-ben-trenn-ung}

\begin{document}
\frontmatter
\maketitle
% impressum
\impressum

% ab hier zeilenabstand 1,4 fach 10pt/14pt
\setstretch{1.4}

\section*{Kurzfassung}
In dieser Dokumentation wird das Spiel 'Abyss' vorgestellt. Das Jump'n'Run Spiel, 
welches in Unity3D entwickelt wurde. Der Platformer entstand im Rahmen der 
Veranstaltung Advanced Mobile Application Engineering. Die Aufgabe war eine mobile 
Applikation über zwei Semester zu entwickeln. 

% inhaltsverzeichnis einfügen
\tableofcontents

\mainmatter
% hier kommen die kapitel der arbeit
\chapter{Einleitung}
\label{cha:einleitung}
In diesem Abschnitt wird 'Abyss' kurz vorgestellt, dabei wird vor allem auf 
die Motivation und das Ziel der Applikation eingegangen.

% Abschnitt: Problemstellung
\section{Problemstellung}
\label{sec:einleitung:problemstellung}
Super Mario Bros. \cite{mario} ist wohl eines der bekanntesten Jump'n'Run Spiele und
der Grundstein für einige weitere Platformer. Trotz der großen Anzahl an Spielen,
die zurzeit auf dem Markt erhältlich ist ist wohl kaum eines so erfolgreich gewesen
wie die, in den der kleine rote Klemptner Mario durch verschiedene Welten rennt. \newline
Da Nintendo erst spät damit angefangen hat die App-Stores für sich zu entdecken 
fehlte vielen Spielern lange Zeit ein guter Platformer für das Smartphone, 
der nicht langweilig wird.


% Abschnitt: Zielsetzung
\section{Zielsetzung}
\label{sec:einleitung:zielsetzung}
Das Ziel ist ein Mobile Game zu entwickeln, dass eine möglichst intuitive Bedienung
hat. Zudem soll das Spiel vielfältig in seiner Funktionalität sein, sodass es 
über einen längeren Zeitraum das Interesse vom Spieler halten kann. \newline
Das beinhaltet nicht nur die grundlegenden Spielfunktionen, sondern auch eine 
Multiplayer-Funktion, die im Zuge der Entwicklung umgesetzt werden soll. 
Die Spielfunktionen sollen in verschiedenen Leveln umgesetzt werden, die sich 
auch im Grunddesign unterscheiden.

% Abschnitt: Struktur der Arbeit
\section{Struktur der Arbeit}
\label{sec:einleitung:struktur}
Nach dieser kurzen Einleitung sollen die Grundlagen für die Arbeit erläutert werden.
Anschließend sollen die Anforderungen herausgearbeitet werden, die später zu einem Konzept 
ergänzt werden sollen. Anschließend wird beschrieben, wie das Konzept umgesetzt wurde, bevor
die Arbeit in einer Zusammenfassung und einem Ausblick beendet wird.
\chapter{Grundlagen}
\label{cha:grundlagen}
Kurzes Text zum Kapitel und Erklärung der Grundlagen.

% Abschnitt: Spielablauf
\section{Spielablauf}
\label{sec:grundlagen:spielablauf}
Spialgrundlagen beschreiben
evtl. Jump'n'Run erklären
...
\chapter{Anforderungen}
\label{cha:anforderungen}
Kurzes Text

% Abschnitt: Funktionale Anforderungen
\section{Funktionale Anforderungen}
\label{sec:grundlagen:funktionaleAnforderungen}

% Abschnitt: Nicht-Funktionale Anforderungen
\section{Nicht-Funktionale Anforderungen}
\label{sec:grundlagen:nichtFunktionaleAnforderungen}
\chapter{Konzeption \& Entwurf}
\label{cha:konzeption}
Kurzer Text

% Abschnitt: Konzept
\section{Konzept}
\label{sec:konzeption:konzept}

% Abschnitt: Prototyping
\section{Prototyping}
\label{sec:konzeption:prototyping}

% Abschnitt: Layout
\section{Layout}
\label{sec:konzeption:layout}
Wie sieht unsere App aus
Features
\chapter{Implemenentierung}
\label{cha:implementierung}
In diesem Kapitel werden zunächst die Technologien besprochen, die für die Umsetzung nötig waren. Anschließend wird ausgeführt wie das Konzept schlussendlich umgesetzt wurde. Dabei wird auch das Design gezeigt und auf Probleme bei der Umsetzung eingangen.

% Abschnitt: Technologien
\section{Technologien}
\label{sec:grundlagen:technologien}
Dieser Abschnitt dient dazu die genutzten Technologien kurz vorzustellen und eventuell auf Vor- und Nachteile sowie auch auf Besonderheiten einzugehen.

\subsection{Unity}
\label{subsec:grundlagen:technologien:unity}
unity
Vor- und Nachteile
wieso Unity gewählt?
Tiles 
Inspector
Prefabs
usw.

\subsection{Weitere Frameworks}
\label{subsec:implementierung:technologien:frameworks}
z.B. Rest, PUN Server, Datenbank

\subsection{Unity \& Mobile Applications}
\label{subsec:implementierung:technologien:mobile}

\section{Umsetzung}
\label{sec:grundlagen:umsetzung}

\subsection{Architektur}
\label{subsec:implementierung:umsetung:architektur}

\subsection{Nutzerprofile}
\label{subsec:implementierung:umsetzung:nutzerprofile}

\subsection{Multiplayer}
\label{subsec:implementierung:umsetzung:multiplayer}

\subsection{Einstellungen}
\label{subsec:implementierung:umsetzung:einstellungen}

\subsection{Level}
\label{subsec:implementierung:umsetzung:level}

\section{Probleme}
\label{sec:implementierung:probleme}
% Design von Assets -> Kostenlose genommen, da aufwändig

\chapter{Zusammenfassung \& Ausblick}
\label{cha:zusammenfassungAusblick}
In diesem letzten Kapitel wird die Arbeit nochmal zusammengefasst um anschließend einen Ausblick für die Zukunft zu geben.

% Abschnitt: Zusammenfassung
\section{Zusammenfassung}
\label{sec:grundlagen:zusammenfassung}
Da Unity als Engine schon viele vordefinierte Funktionen anbietet, darunter zum Beispiel die Collision-Detection, ist das Arbeiten zunächst ungewohnt im Gegensatz zur Arbeit mit einer Entwicklungsumgebung wie Android Studio. Ein Grund dafür ist, das ein großer Anteil nicht über Programmcode, sondern über die GUI-Oberfläche die Unity zur Verfügung stellt realisiert wird. Der Inspector des Editors lässt die Componenten der Gameobjects mit den jeweiligen Attributen schnell verändern. Dadurch ist das Grundgerüst des Spiels zwar nur sehr aufwendig zu realisieren, aber später auch sehr wandlungsfähig und kann im Nachhinein nach belieben im Detail angepasst werden. 

Nach dem Fertigstellen der App kann durch die gute Kompatibilität von Unity mit sehr vielen Plattformen das Spiel nicht nur auf Android, sondern auf weiteren Plattformen verwendet werden. Dadurch, und durch die Verwendung von Photon, wird unsere Abyss-Anwendung zu einem \textit{Crossplatformgame}.

Das Verwenden von eigenen UI-Elementen sollte man als Design-Anfänger outsourcen, da das Erstellen von Bilder, die visuell ansprechend sein sollen, eine künstlerische Ader erfordert. Das gleiche gilt für die Erstellung von eigenen Soundeffekten und Hintergrundmusik. Man kann durch die Verwendung von eigenen Materialien das Spiel im hohem Maß individualisieren, doch ist der Aufwand, der hinter dieser Arbeit steckt, nicht zu unterschätzen. 

Das Projektmanagement der Unity-Anwendung ist mit Git fehleranfällig. Erstellt man jedoch für jedes Problem einen eigenen Branch, und versucht diese und damit das Mergen so klein wie möglich zu halten, können die entstehenden Probleme minimal gehalten werden.

% Abschnitt: Ausblick
\section{Ausblick}
\label{sec:grundlagen:ausblick}
Nachdem das Grundgerüst des Spiels mit den selbstentworfenen Funktionen und Assets erstellt wurde, ist das Spiel schnell und simpel erweiterbar. Durch das Verwenden von Prefabs und den vorhandenen Tilemaps können in kürze Spielflächen erstellt und Hindernisse hinzugefügt werden. Um neue Spielobjekte zu erstellen, können die vorhandenen Skripte angefügt und durch neue, aber simple Skripte erweitert werden. Dadurch kann das Spiel nicht nur an Leveln, sondern auch an Spielkomplexität wachsen.

% Bibliograhpy
\bibliographystyle{splncs}
\begingroup
\interlinepenalty 10000
\sloppy
\bibliography{literature}
\endgroup

% anhänge
\appendix
% hier kommen die anhänge
\input{sources}

\backmatter			% abtrennung für verzeichnisse

% hier die verzeichnisse
\listoffigures

\end{document}
