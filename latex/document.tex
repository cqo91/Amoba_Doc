\documentclass[numbers=noenddot]{thesis}

% hier namen etc. einsetzen
\fullname{Lukas Hennig\\Felix Rottler\\Natalie Spister}
\headline{Advanced Mobile Business Applications\\Dokumentation Abyss}
\jahr{2018}
\gutachterA{Marc Schickler}
\typ{Anwendungsfach}
\fakultaet{Ingenieurwissenschaften, Informatik und \\Psychologie}
\institut{Institut für Datenbanken und Informationssysteme}

% Falls keine Lizenz gewünscht wird bitte den folgenden Text entfernen.
% Die Lizenz erlaubt es zu nichtkommerziellen Zwecken die Arbeit zu
% vervielfältigen und Kopien zu machen. Dabei muss aber immer der Autor
% angegeben werden. Eine kommerzielle Verwertung ist für den Autor
% weiter möglich.
\license{
This work is licensed under the Creative Commons.
Attribution-NonCommercial-ShareAlike 3.0 License. To view a copy of this
license, visit http://creativecommons.org/licenses/by-nc-sa/3.0/de/ or send a
letter to Creative Commons, 543 Howard Street, 5th Floor, San Francisco,
California, 94105, USA. \\ Satz: PDF-\LaTeXe
}

\hypersetup{%
	pdftitle=\pdfescapestring{\thetitel},
	pdfauthor={\thefullname},
 	pdfsubject={\thetyp},
}


% trennungsregeln
\hyphenation{Sil-ben-trenn-ung}

\begin{document}
\frontmatter
\maketitle
% impressum
\clearpage
\impressum

\cleardoublepage
% ab hier zeilenabstand 1,4 fach 10pt/14pt
\setstretch{1.4}

\section*{Kurzfassung}
Im Rahmen der Veranstaltung Advanced Mobile Application Engineering 

Die Kurzfassung (engl. Abstract) einer Abschlussarbeit enthält zwei Blöcke. Der
erste Block enthält eine  kurze Hinführung/Motivation zum Thema sowie einer
anschließenden Beschreibung der Problemstellung (ca. 5-8 Sätze). Der zweite
Block der Kursfassung gibt die Zielsetzung bzw. den Beitrag der Abschlussarbeit
wieder (ebenfalls ca. 5-8 Sätze).

===========================================

ChangeLog:

2015-10-12: Hacks für Literaturverzeichnis eingebaut. Kommentare in der BibTex
Datei beachten!

2015-07-21: Fakultätname angepasst (+ Psychologie)



\cleardoublepage

% inhaltsverzeichnis einfügen
\tableofcontents

\mainmatter
% hier kommen die kapitel der arbeit
% % %
%
%	- Einleitung- 
%
%	Ziel:	Gib eine schöne Einleitung mit Motivation, Problemstellung, Zielsetzung und Struktur der Arbeit
%
%	Status: alpha
%
% % %
\chapter{Einleitung}
\label{cha:einleitung}
Hier kommt die Einleitung mit Motivation hin.

% Abschnitt: Problemstellung
\section{Problemstellung}
\label{sec:einleitung:problemstellung}
Beschreibe in diesem Abschnitt die Problemstellung der Arbeit!

% Abschnitt: Zielsetzung
\section{Zielsetzung}
\label{sec:einleitung:zielsetzung}
Beschreibe in diesem Abschnitt die Zielsetzung der Arbeit!

% Abschnitt: Struktur der Arbeit
\section{Struktur der Arbeit}
\label{sec:einleitung:struktur}
Beschreibe in diesem Abschnitt die Struktur der Arbeit!


% Bibliograhpy
\bibliographystyle{splncs}
\begingroup
\interlinepenalty 10000
\sloppy
\bibliography{literature}
\endgroup

% anhänge
\appendix
% hier kommen die anhänge
\chapter{Quelltexte}

In diesem Anhang sind einige wichtige Quelltexte aufgeführt.

\begin{lstlisting}[caption={Zeilencode}]
public class Hello {
    public static void main(String[] args) {
        System.out.println("Hello World");
    }
}
\end{lstlisting}


\backmatter			% abtrennung für verzeichnisse

% hier die verzeichnisse
\listoffigures
\listoftables


\clearpage
\erklaerung

\end{document}
