% Steuerung Anfoderung

\chapter{Anforderungen}
\label{cha:anforderungen}
In diesem Kapitel werden die Anforderungen an das Spiel definiert. Dabei legen die funktionalen Anforderungen fest, wozu das Spiel imstande sein soll. Die nicht-funktionalen Anforderungen beschreiben, wie gut diese Anforderungen umgesetzt werden sollen.

% Abschnitt: Funktionale Anforderungen
\section{Funktionale Anforderungen}
\label{sec:grundlagen:funktionaleAnforderungen}
In Tabelle \ref{tab:grundlagen:funktionaleAnforderungen} werden die funktionalen Anforderungen an das Spiel und die jeweilige Priorisierung mit Werten von 1 (unwichtig) bis 5 (unabdingbar) formuliert.

\begin{center}
    \label{tab:grundlagen:funktionaleAnforderungen}
    \begin{tabular}{ c | l | c}
        Abkürzung & Anforderung & Priorität\\
        \hline
        FA01 & An- und Abmelden & 4 \\
        \hline
        FA02 & Spiel erstellen & 5 \\
        \hline
        FA03 & Spiel beitreten & 5 \\
        \hline
        FA04 & Multiplayer (2 Spieler) & 5 \\
        \hline
        FA05 & Animation der Spieler & 4 \\
        \hline
        FA06 & Verschiedene Level & 4 \\
        \hline
        FA07 & Verschiedene Level-Themen & 3 \\
        \hline
        FA08 & Verschiedene Spielobjekte & 4 \\
        \hline
        FA09 & Animation der Spielobjekte & 2 \\
        \hline
        FA10 & Simple Spielsteuerung & 5 \\
    \end{tabular}
    \captionof{table}{Funktionale Anforderungen} 
\end{center}

\textbf{FA01 An- und Abmelden}\\*
Der Nutzer kann sich im Spiel anmelden. Anschließend kann er das Spiel uneingeschränkt verwenden. Er bleibt so lange angemeldet bis er sich abmeldet.

\textbf{FA02 Spiel erstellen}\\*
Ein Spiel kann von jedem angemeldeten Nutzer erstellt werden. Dabei kann der Nutzer nur die Level verwenden, die er freigeschalten hat.

\textbf{FA03 Spiel beitreten}\\*
Jeder Nutzer kann jedem beliebigen Spiel beitreten unabhängig davon welche Level der Nutzer selbst freigeschalten hat.

\textbf{FA04 Multiplayer}\\*
Das Spiel soll ausschließlich von zwei Nutzern gespielt werden können. Die Nutzer sollen dabei den jeweiligen Mitspieler auf ihrem eigenen Gerät sehen können. Zudem soll teilweise ein kooperatives Spiel zwischen den Spielern erzwingbar sein.

\textbf{FA05 Animation der Spieler}\\*
Die verschiedenen Möglichkeiten der Steuerung eines Spielers sollen mit Animationen unterlegt werden. 

\textbf{FA06 Verschiedene Level}\\*
Es soll mehr als ein Level geben. Der Nutzer hat anfangs nur die Möglichkeit das erste Level zu spielen. Erst nach Abschluss eines Levels wird das nächste freigeschalten.

\textbf{FA07 Verschiedene Level-Themen}\\*
Die Level sollen sich im Design voneinander unterscheiden. Die Spielerfiguren und Spielobjekte bleiben dabei die gleichen im Aussehen und in ihrer Funktion.

\textbf{FA08 Verschiedene Spielobjekte}\\*
Durch verschiedene Spielobjekte innherhalb des Spiels wird das Spiel abwechslungsreich und komplex. Jedes dieser Spielobjekte hat dabei einen unterschiedlichen Effekt.

\textbf{FA09 Animation der Spielobjekte}\\*
Die Spielobjekte sollen durch Animationen als solche hervorgehoben werden. Dabei können sie bei Aktivierung verschiedene Effekte/Animationen/Geräusche machen, die den Zweck des Spielobjekts entprechen.

\textbf{FA10 Simple Spielsteuerung}\\*
Das Spiel soll möglichst einfach gesteuert werden können, d.h. auch das Eingaben, die möglicherweise falsch interpretiert werden können (Wischen in verschiedene Richtungen oder langes Drücken), sollen möglichst vermieden werden.


% Abschnitt: Nicht-Funktionale Anforderungen
\section{Nicht-Funktionale Anforderungen}
\label{sec:grundlagen:nichtFunktionaleAnforderungen}
Die in Kapitel \ref{sec:grundlagen:funktionaleAnforderungen} beschriebenen funktionale Anforderungen bestimmen die Funktionen und Umfang des Spiels. In der Tabelle \ref{tab:grundlagen:nichtFunktionaleAnforderungen} werden nun die nichtfunktionalen Anforderungen dargelegt, die die Qualität der Umsetzung festlegen.

\begin{center}
    \label{tab:grundlagen:nichtFunktionaleAnforderungen}
    \begin{tabular}{ c | l | c}
        Abkürzung & Anforderung & Priorität\\
        \hline
        NFA01 & Zuverlässigkeit & 5 \\
        \hline
        NFA02 & Aussehen und Handhabung & 5 \\
        \hline
        NFA03 & Benutzbarkeit & 4 \\
        \hline
        NFA04 & Funktionalität & 5 \\
        \hline
        NFA05 & Effizienz & 3 \\
    \end{tabular}
    \captionof{table}{Nichtfunktionale Anforderungen} 
\end{center}

\textbf{NFA01 Zuverlässigkeit}\\*
Das Spiel ist konsistent in der Ausführung. Es bricht nicht unerwartet ab und jedes Level kann von beiden Spielern beendet werden.

\textbf{NFA02 Aussehen und Handhabung}\\*
Das Spiel hinterlässt einen einen guten optischen Eindruck und besitzt eine uintuitive Steuerung.

\textbf{NFA03 Benutzbarkeit}\\*
Der Ablauf des Spiels ist leicht ersichtlich, sowohl im Menü als auch in den jeweiligen Level.

\textbf{NFA04 Funktionalität}\\*
Die in Kapitel \ref{sec:grundlagen:funktionaleAnforderungen} vorgestellten Anforderungen werden abhängig von ihrer Priorisierung erfolgreich umgesetzt.

\textbf{NFA05 Effizienz}\\*
Das Spiel soll auf dem eigenen als auch auf dem Gerät des Mitspielers flüssig laufen. 



