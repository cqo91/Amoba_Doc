\chapter{Implemenentierung}
\label{cha:implementierung}
In diesem Kapitel werden zunächst die Technologien besprochen, die für die Umsetzung nötig waren. Anschließend wird ausgeführt wie das Konzept schlussendlich umgesetzt wurde. Dabei wird auch das Design gezeigt und auf Probleme bei der Umsetzung eingangen.

% Abschnitt: Technologien
\section{Technologien}
\label{sec:grundlagen:technologien}
Dieser Abschnitt dient dazu die genutzten Technologien kurz vorzustellen und eventuell auf Vor- und Nachteile sowie auch auf Besonderheiten einzugehen.

\subsection{Unity}
\label{subsec:grundlagen:technologien:unity}
unity
Vor- und Nachteile
wieso Unity gewählt?
Tiles 
Inspector
Prefabs
usw.

\subsection{Weitere Frameworks}
\label{subsec:implementierung:technologien:frameworks}
z.B. Rest, PUN Server, Datenbank

\subsection{Unity \& Mobile Applications}
\label{subsec:implementierung:technologien:mobile}

\section{Umsetzung}
\label{sec:grundlagen:umsetzung}

\subsection{Architektur}
\label{subsec:implementierung:umsetung:architektur}

\subsection{Nutzerprofile}
\label{subsec:implementierung:umsetzung:nutzerprofile}

\subsection{Multiplayer}
\label{subsec:implementierung:umsetzung:multiplayer}

\subsection{Einstellungen}
\label{subsec:implementierung:umsetzung:einstellungen}

\subsection{Level}
\label{subsec:implementierung:umsetzung:level}

\section{Probleme}
\label{sec:implementierung:probleme}
% Design von Assets -> Kostenlose genommen, da aufwändig
